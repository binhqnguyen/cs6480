\section{Introduction}

As the cellular network is moving toward all-IP, accessing a variety of services 
using smart phones or tablets has become more and more popular. 
While the popular web browsing does not require a strict Quality 
of Service (QoS) requirement, services such as gaming and cloud services 
require a certain delay and bandwidth budget in order to operate appropriately. These QoS realizations are deeply 
integrated into the current cellular network design as multiple QCIs (QoS class of identifier) are specified in 3GPP specifications. 
For example, as in ~\cite{3GPP:2012}, online real time gaming services require 50ms delay budget with 
a guaranteed bit-rate (GBR), interactive gaming services require 100ms of delay budget.\\

The current cellular network architecture, however, has some limitations that hinder the efforts to 
deeply reduce network delay. As mentioned in ~\cite{balakrishnan2009s}, the number of gateways in 
3G networks is small, and since these gateways are the Internet access points, 
traffic in cellular network have to pass through these geographically distributed nodes and 
this adds additional delay to the path as opposed to direct routing 
(hierarchical routing issue ~\cite{Dong:2011:MIU:2043468.2043472}). 
Moreover, despite the fact that the recent network design (e.i., 4G/Longterm Evolution and Evolved Packet Core or 
4G/LTE-EPC) is flat and the delay caused by the cellular 
core network is reduced (the average end-to-end RTT is about 70ms in 4G/LTE ~\cite{huang2013depth}), 
end-to-end delay still suffers from and is gradually dominated by the Internet delay (the delay between 
cellular gateways and Internet servers). This delay could not be eliminated by any mean
due to the constraint of speed of light.\\

There are approaches that try to reduce the end-to-end delay in cellular networks by 
leveraging Content Delivery Network (CDN) and knowledge about the geographical distribution of 
cellular gateways. For example, cloud providers move their CDN servers close to the cellular 
core network to reduce the Internet delay portion; knowledge about the geographical distribution 
of IP addresses of mobile devices aids the CDN servers placement ~\cite{xu2011cellular}; or 
grouping P2P (peer-to-peer) devices that are geographically closed could reduce the 
delay between them ~\cite{agarwal2009matchmaking}. All of the above works, 
however, concentrate on the Internet portion of the end-to-end delay yet 
can not further reduce the total delay due to the hierarchical routing issue remained inside the core. 
Moreover, the fact that the number of cellular gateways is small and these gateways are geographically distributed 
adds another degree of complexity to the above approaches.\\

Our work is strongly inspired by MOCA ~\cite{banerjee2013moca}. 
MOCA proposed an offloading architecture for cellular networks in which 
services can be placed inside the core network to eliminate both the hierarchical routing issue 
and Internet delay. MOCA leverages SDN (Software-defined Network) to offload traffic 
to a local SPGW (service and packet gateway) instance and an offloading server inside the core.  
In MOCA architecture, although modifications in the current network were intentionally minimized, 
the MME (Mobility Management Entity) still played a centralized role and need to be modified. 
Here, we raise the question of whether SDN could realize the offloading architecture 
proposed in MOCA without touching the components of the current cellular architecture.\\

The rest of the paper is presented as follow: section 2 provides a background information 
of 4G/LTE-EPC networks, section 3 describes our approach and the high-level system design, 
section 4 presents our implementation, evaluation and results are presented in section 5.\\




